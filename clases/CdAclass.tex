\documentclass{article}



\begin{document}

% Formatando a seção de título do artigo.
	
\titulo{Interpretacion de Modelos Sismicos en Marte y La Luna} 
% Título do artigo

\autor{Nome do Autor} % Autor
\instituto{Instituição do Autor} % Instituto: Utilizar somente o nome da instituição, por exemplo, Universidade..., Instituto Federal..., Escola Estadual...
% Maiores especificações, como departamento, centro e etc, podem ser adicionadas na sessão Sobre o Autor.

% Para mais autores basta ir adicionando com o comando \autor{}.
% Por exemplo
%\autor{Primeiro}
%\autor{Segundo}
%\autor{Terceiro}
%\instituto{Universidade...}

% Se os autores possuem instituições diferentes, utilize \autor[n]{...} e \instituto[n]{...}.
% Exemplo:
%\autor[1]{Primeiro}
%\autor[1]{Segundo}
%\autor[2]{Terceiro}
%\instituto[1]{Universidade...}
%\instituto[2]{Universidade...}

% Resumo (em português)
\resumo{Este é um exemplo de artigo construído com a classe CdA (Latex). Esta classe foi criada para atender as necessidades da revista Cadernos de Astronomia. Detalhamos como utilizar cada comando específico da classe, orientando os autores na preparação de seus trabalhos para submissões.}

% Abstract (em inglês).
\abstract{This is a example of article written using the CdA class (Latex). This class was created to meet the needs of Cadernos de Astronomia magazine. We detail how to use each specific command of the class, guiding authors in preparing their work for submissions.}

\pchave{astronomia, cadernos, exemplo.}			 % Palavras-chave (separadas por vírgulas)
\keywords{astronomy, notebooks, example.} 	 % Palavras-chave traduzidas para o inglês

\maketitle  % Comando para compor o título do artigo.

\oautor{P. Autor}  				% Para o cabeçalho: Adicione o nome do autor com as iniciais abreviadas.

%\oautor{P. Autor et tal   % Em caso de dois ou mais autores, inclua somente o primeiro autor seguido de  "et al" (sem as aspas).

% Sessões e subsseções são acionadas de forma normal com os comandos \section{} e \subsection{}.
\section{Introdução}
A classe \textit{cda} é baseada na classe \textit{article} com alguns novos comandos adicionados para se adequar à formatação da revista Cadernos de Astronomia (CdA).

O presente texto pretende ser um exemplo e guia para autores sobre como redigir um texto para publicação nos Cadernos de Astronomia utilizando LaTex. O corpo editorial dos CdA recomenda, fortemente, que os artigos submetidos sejam redigidos em LaTex, o que deve acelerar o processo de editoração e a definitiva publicação, além de evitar possíveis incompatibilidades gráficas.\footnote{Os CdA também aceita artigos no formato Writer (Libre Ofiice) e Word.}

O autor que nunca teve contato com o ambiente LaTex e deseja ter mais detalhes pode verificar o vídeo \href{https://youtu.be/T9Q64dAzW20}{Como instalar TexLive Latex} para instruções básicas sobre como fazer a instalação da distribuição de LaTex chamada TexLive. Para um introdução rápida sobre como redigir documentos usando o LaTex pode-se verificar os tutoriais do site \href{http://latexbr.blogspot.com/2012/09/bibliografia-com-bibtex.html}{LaTex BR}. 

O LaTex é totalmente gratuito e muito material instrutivo pode ser encontrado na internet e em bibliotecas. É uma ferramenta profissional para edição de textos no meio acadêmico. Por isso recomendamos sua utilização.

\subsection{Pacotes automáticos}

Os pacotes \textit{[T1]\{fontenc\}, [utf8]\{inputenc\}, [brazil]\{babel\}} são automaticamente utilizados pela classe, necessários para a escrita de textos em português (Brasil).

Os pacotes \textit{\{amsfonts\}, \{amsmath\}, \{amssymb\}} são também abertos automaticamente, sendo úteis para a composição de equações matemáticas (são os mesmos usados na classe Revtex4).

Também são acionados os pacotes \textit{\{graphicx\}} e \textit{\{float\}} para a inserção de figuras e tabelas.

\subsection{Adicionando mais pacotes}
Qualquer outro pacote que os autores necessitem podem ser adicionados normalmente no preâmbulo do arquivo através do comando \verb|\usepackage{Nome do pacote}|.

\section{Título, autores, institutos, resumo e abstract}
Para a composição do título há alguns comandos próprios. Utiliza-se \verb|\titulo\{...\}| para inserir o título do artigo. Há ainda os comandos \verb|\autor{Nome do autor}| para o autor e \verb|\instituto{Nome do instituto}| para a afiliação do autor. 

%\footnote{...} É o comando para adicionar notas de rodapé ao texto.

Para o caso de haver mais de um autor, basta ir adicionando os demais com o mesmo comando. Por exemplo:
\begin{verbatim}
\autor{Primeiro Autor}
\autor{Segundo Autor}
\autor{Terceiro Autor}
\instituto{Nome do instituto}
\end{verbatim}
\textbf{Importante:} No campo para inserir o instituto, informar apenas o nome principal (Universidade, Escola e etc), qualquer informação de repartições internas (como departamento, centro e etc) somente deve ser incluída na sessão \textbf{\textit{Sobre o autor}}. Caso a instituição seja estrangeira, adicione também a cidade e o país (ex: Columbia University, Nova York, EUA).
	
Se os autores possuem instituições diferentes, utilize \verb|\autor[n]{Nome do autor}| e \verb|\instituto[n]{Nome do instituto}| para ser adicionado os índices no estilo nota de rodapé, onde \verb|n| deve ser um número (1,2,...). Exemplo:
\begin{verbatim}
\autor[1]{Primeiro Autor}
\autor[1]{Segundo Autor}
\autor[2]{Terceiro Autor}
\instituto[1]{Nome do instituto 1}
\instituto[2]{Nome do instituto 2}
\end{verbatim}
	
Os comandos \verb|\resumo{...}| e \verb|\abstract{...}| são utilizados para a inserção do resumo e o abstract (em inglês) do artigo.

Use o comando \verb|\oautor{P. Autor}| para definir o autor no cabeçalho do artigo. Deve ser utilizado as iniciais abreviadas, por exemplo P. Autor. Caso haja mais de um autor, utilize et al, como P. Autor et al.

Por fim, palavras-chave são adicionadas com \verb|\pchave{...}|, enquanto que, para o equivalente em inglês (keywords) usamos \verb|\keywords{...}|. Cada palavra chave deve ser separada da seguinte por uma vírgula. Letras maiúsculas devem ser usadas somente quando envolver nome próprio.

Por fim, o comando \verb|\maketitle| compõe o bloco de título do artigo.


\section{Comandos comuns à classe article}
Os demais comando necessários são os mesmos da classe \textit{article}. Para iniciar novas seções e subseções, veja abaixo:

\begin{verbatim}
\section{Título da seção}
Texto da sessão. Quando um parágrafo
quiser ser iniciado, basta deixar uma
linha em branco.

Este é um novo parágrafo da sessão.

\subsection{Título da subseção}
Este é um exemplo de como começar
uma subseção.
\end{verbatim}

Além disso, os ambientes para adicionar equações, figuras e tabelas também permanecem os mesmos.

\subsection{Exemplo de equação}
Recomenda-se que todas equações inseridas no artigo sejam numeradas. Por isso, para um única equação pode-se utilizar o ambiente \textit{equation}. A seguinte equação,
\begin{equation}\label{eq:1}
 R_{\mu\nu} - \frac{1}{2}\,Rg_{\mu\nu}=\kappa T_{\mu\nu},
\end{equation}
é produzida pelo código abaixo,
\begin{verbatim}
\begin{equation}\label{eq:1}
R_{\mu\nu} - \frac{1}{2}
Rg_{\mu\nu}=\kappa T_{\mu\nu},
\end{equation}
\end{verbatim}
Após abrir o ambiente \textit{equation}, usamos \verb|\label{...}| para obter uma forma de referenciar esta equação ao longo do texto. Isso é feito com o comando \verb|\eqref{}|. Por exemplo, a equação acima pode ser citada usando \verb|\eqref{eq:1}|, o que produzirá \eqref{eq:1}. O argumento de \verb|\label| pode ser qualquer coisa, sendo prático que sejam escolhidos de forma a facilitar a identificação.

\subsection{Exemplo de figura}
A forma mais comum de se adicionar uma figura ao texto é por meio do ambiente \textit{figure}.
\begin{figure}[h]
	\centering
	\includegraphics[width=\columnwidth]{figCdA}
	\caption{Legenda da figura.}
	\label{fig:1}
\end{figure}

\begin{figure*}[t]
	\centering
	\includegraphics[width=.7\linewidth]{figCdA}
	\caption{Legenda da figura.}
	\label{fig:2}
\end{figure*}

A \figref{fig:1} foi inserida com o seguinte código,
\begin{verbatim}
	\begin{figure}[t]
	\centering
	\includegraphics[width=.9\linewidth]{figCdA}
	\caption{Legenda da figura.}
	\label{fig:1}
	\end{figure}
\end{verbatim}
O ambiente \textit{figure} é iniciado com a opção {\tt [t]} que indica que a figura deve ser inserida no topo da página. O comando \verb|\centering| centraliza a imagem inserida na coluna. O arquivo com a imagem é acionado através do comando \verb|\includegraphics|, onde a opção \verb|[width=.9\columnwidth]| diz que a figura deve ter, no máximo $0.9$ vezes o tamanho da coluna e \verb|{figCdA}| é o nome do arquivo acionado. Note que a figura deve estar na mesma pasta onde o arquivo \textit{.tex} está sendo escrito. Por fim, adiciona-se uma legenda com o comando \verb|\caption{...}| e um rótulo com \verb|\label{...}|, para posteriormente referenciar a figura através do comando \verb|\figref{...}|.


Se a figura for muito grande para o tamanho da coluna, pode-se utilizar o ambiente \verb|\begin{figure*}...\end{figure*}|, que automaticamente ajusta a figura para ocupar toda a área de texto, como na \figref{fig:2}. O conteúdo do ambiente continua sendo o mesmo do exemplo anterior, com exceção da opção que determina a largura máxima da figura, onde agora podemos usar \verb|width=0.8\linewidth|, já que não estamos restrito ao tamanho da coluna.

\section{Examplo de tabela}
Tabelas são adicionadas com o ambiente \textit{tabular} dentro do ambiente \textit{table} para produzir um resultado centralizado na coluna. O código a seguir é um exemplo simples de tabela.
\begin{verbatim}
	\begin{table}[h]
	\centering
	\caption{Legenda da tabela}
	\label{tab:1}
	\begin{tabular}{ |l|c|r| }
	\hline
	\multicolumn{3}{|c|}{Linha mesclada}\\
	\hline
	Coluna 1 & Coluna 2 & Coluna 3 \\
	\hline
	cel1   & AF    &AFG\\
	cel2 &   AX  & ALA  \\
	cel3 &AL & ALB\\
	cel3  &DZ & DZA\\
	cel4 &   AS  & ASM\\
	\hline
	\end{tabular}
	\end{table}
\end{verbatim}

O resultado pode ser visto logo abaixo, na \tabref{tab:1}. Muitos comando possuem o mesmo significado do que foi dito na subseção anterior, sobre figuras. A grande diferença está no ambiente \textit{tabular} e seu conteúdo, que constrói a tabela propriamente dita. Seu argumento \{|\verb|l||\verb|c||\verb|r||\}, indica que são 3 colunas, a primeira alinhada à esquerda (l), a segunda centralizada (c) e a terceira alinhada à direita (r). O comando \verb|\hline| insere uma linha horizontal e \verb|&| separa cada célula de uma linha. Para ir para a próxima linha usamos o comando de quebra de linha \verb|\\|.
\begin{table}[h]
        \renewcommand{\arraystretch}{1.2}
	\centering
	\caption{Legenda da tabela}
	\label{tab:1}
\begin{tabular}{ l|c|r }
	\hline\hline
	\multicolumn{3}{c}{Linha mesclada}\\
	\hline
	Coluna 1 & Coluna 2 & Coluna 3 \\
	\hline
	cel1   & AF    &AFG\\
	cel2 &   AX  & ALA  \\
	cel3 &AL & ALB\\
	cel3  &DZ & DZA\\
	cel4 &   AS  & ASM\\
	\hline\hline
\end{tabular}
\end{table}

O comando \verb|\multicolumn{cols}{pos}{text}| inseri uma linha mesclada. \verb|cols| indica quantas células estão sendo unidas, \verb|pos| o alinhamento e \verb|text| é o texto contido na célula mesclada.

\section{Sessão agradecimentos}
Quando necessário, uma sessão de agradecimentos pode ser adicionada ao final do artigo (após as conclusões). Há um ambiente específico para isso,
\begin{verbatim}
\begin{agradecimentos}
Aqui entra o texto de agradecimentos.
\end{agradecimentos}
\end{verbatim}

\appendix

\section{Sessão de apêndice}
Caso seja necessário adicionar sessões de apêndice ao artigo, o mesmo deve ser feito declarando o comando \verb|\appendix|. A partir daí, toda vez que uma nova seção é inserida, tem-se um apêndice.

\begin{sobreautor}[m]
Os Cadernos de Astronomia exigem uma breve descrição dos autores ao final do artigo, antes das referências. Deve ser feita dentro do ambiente \textit{sobreautor}, como no exemplo abaixo. \textbf{É imprescindível que seja adicionado um e-mail para contato.}
\begin{verbatim}
\begin{sobreautor}[opção]
P. Autor (autor@email...) é especialista
em...
\end{sobreautor}
\end{verbatim}

O campo \verb|[opção]| serve para mudar o título padrão da sessão de acordo com gênero e número, conforme a \tabref{tab:2}.
\begin{table}[h!]
    \renewcommand{\arraystretch}{1.2}
	\centering
	\caption{Lista de opções do ambiente \textit{sobreautor}.}
	\label{tab:2}
	\begin{tabular}{c|l}
		\hline\hline
		Opção & Titulo da sessão \\
		\hline
		m & Sobre o autor \\
		\hline
		mp & Sobre os autores \\
		\hline
		f & Sobre a autora \\
		\hline
		fp & Sobre as autoras \\
		\hline\hline
	\end{tabular}
\end{table}

A opção padrão é \textit{m}, ou seja, no caso de um único autor, basta não inserir nenhuma opção no ambiente.
\end{sobreautor}

\section*{Inserindo referências bibliográficas}
Se a bibliografia utilizada estiver armazenada em um arquivo \textit{.bib} (BibTEX) a classe CdA automaticamente configura a sessão de referências no formato adequado para a revista. Basta que o arquivo \textit{.bib} esteja na mesma pasta do arquivo \textit{.tex}, que o comando \verb|\bibliography{arquivo.bib}| produzirá a seção \textit{\textbf{Referências}}. O site \href{http://latexbr.blogspot.com/2012/09/bibliografia-com-bibtex.html}{LaTex BR} apresenta uma boa introdução sobre como administrar bibliografia com o BibTEX e como usá-lo em LaTex.

No corpo do texto, as referências são citadas usando numeração, através do comando \verb|\cite{citekey}|, onde \verb|citekey| é a identificação da referência requisitada. Caso a bibliografia seja inserida manualmente, deve-se utilizar o ambiente \textit{thebibliography}, como no exemplo abaixo.
\begin{verbatim}
	\begin{thebibliography}{99}
	\bibitem{citekey} P. Autor e S. Autor,
	{\it Título do artigo},	Nome abreviado
	da revista/jornal {\bf vol} (número),
	página inicial (ano).
	\end{thebibliography}
\end{verbatim}
A opção \verb|{99}| serve para configurar parâmetros geométricos do ambiente criado. Na prática, ele define as margens da lista de referências criada de acordo com o número de dígitos possíveis. No caso acima, está adequado para até dois dígitos, ou seja, se o artigo tiver mais de 99 referências, muda-se para \verb|{999}|. O comando \verb|\bibitem{citekey}| é quem introduz a bibliografia associada ao rótulo \verb|citekey|.
Abaixo, apresentamos instruções e exemplos de como incluir referências de acordo com o tipo de publicação que está sendo citada:\\
Artigos em periódico: \cite{CdA:art,Fabris:2009wa}\\
Livro: \cite{CdA:livro,Weinberg1972gcpa}\\
Capítulo de Livro: \cite{CdA:proc,Toniato:2016sma}\\
Teses: \cite{CdA:tese,Clifton:2006jh}\\
Preprint: \cite{CdA:preprint,jimnez2020anisotropic}


\bibliography{Bibliografia}{} 

%\begin{thebibliography}{10}
%	
%	\bibitem{CdA:art}
%	P.~Autor \protect\BIBand{} S.~Autor, \emph{T{\'\i}tulo do artigo}, Nome
%	abreviado do Jornal \textbf{Vol}~(n{\'u}mero), p{\'a}gina inicial--p{\'a}gina
%	final (Ano).
%	
%	\bibitem{Fabris:2009wa}
%	J.~Fabris \protect\BIBand{} H.~Velten, \emph{{MOND virial theorem applied to a
%			galaxy cluster}}, Braz.J.Phys. \textbf{39}, 592--595 (2009).
%	
%	\bibitem{CdA:livro}
%	P.~Autor, \emph{T{\'\i}tulo do livro} (Editora, Cidade, Ano).
%	
%	\bibitem{Weinberg1972gcpa}
%	S.~{Weinberg}, \emph{{Gravitation and cosmology: principles and applications of
%			the general theory of relativity}} (John Wiley \& Sons, New York, 1972).
%	
%	\bibitem{CdA:proc}
%	P.~Autor, \emph{T{\'\i}tulo do cap{\'\i}tulo}, in \emph{T{\'\i}tulo do livro ou
%		anais}, editado por Editores (Editora, Cidade, Ano), pag.inicial--pag.final.
%	
%	\bibitem{Toniato:2016sma}
%	J.~D. Toniato, \emph{{Geometric scalar theory of gravity}}, in \emph{{The
%			Cosmic Microwave Background}}, editado por J.~C. Fabris, O.~F. Piattella,
%	D.~C. Rodrigues, H.~Velten \protect\BIBand{} W.~Zimdahl ({Springer
%		International Publishing}, 2016), 359--369.
%	
%	\bibitem{CdA:tese}
%	P.~Autor, \emph{T{\'\i}tulo da tese}, Tese de Mestrado, Institui{\c c}{\~a}o
%	(Ano).
%	
%	\bibitem{Clifton:2006jh}
%	T.~Clifton, \emph{{Alternative theories of gravity}}, Tese de Mestrado, King's
%	Coll. London (2006).
%	
%	\bibitem{CdA:preprint}
%	P.~Autor \protect\BIBand{} S.~Autor, \emph{T{\'\i}tulo do artigo},
%	arxiv:n{\'u}mero (Ano).
%	
%	\bibitem{jimnez2020anisotropic}
%	J.~B. Jim{\'e}nez, D.~de~Andr{\'e}s \protect\BIBand{} A.~Delhom,
%	\emph{Anisotropic deformations in a class of projectively-invariant
%		metric-affine theories of gravity}, arXiv:2006.07406 (2020).
%	
%\end{thebibliography}



\end{document}